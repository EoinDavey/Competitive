\documentclass[twocolumn]{article}
\usepackage{fancyhdr}
\usepackage{verbatim}
\usepackage{amsmath}
\usepackage{listings}
\lstset
{
    numbers=left,
    basicstyle=\footnotesize,
    breaklines=true,
}
\usepackage[landscape, a4paper,margin=0.5in]{geometry}
\setcounter{tocdepth}{4}
\setcounter{secnumdepth}{1}

\pagestyle{fancy}
\fancyhf{}
\rhead{\thepage}
\lhead{Maynooth University - Computational Triumph}
\begin{document}
    \title{Data Structures and Algorithms for Competitive Programming}
    \author{Eoin Davey}
    \maketitle
    \tableofcontents
    \newpage

    \section{Graphs}

    \subsection{Search}
        \subsubsection{BFS}
        \lstinputlisting[language=c++]{BFS.cpp}
        \subsubsection{Dijkstras}
        \lstinputlisting[language=c++]{Dijkstras.cpp}
        \subsubsection{A*}
        \lstinputlisting[language=c++]{A_Star.cpp}
        \subsubsection{Bidirectional BFS}
        \lstinputlisting[language=c++]{Bidir_BFS.cpp}

    \subsection{Trees}
        \subsubsection{MST}
        \lstinputlisting[language=c++]{MST.cpp}
        \subsubsection{LCA}
        \lstinputlisting[language=c++]{LCA.cpp}
        \subsubsection{Centroid Decomposition}
        \lstinputlisting[language=c++]{Centroid_Decomp.cpp}

    \subsection{SCC Tarjans}
    \lstinputlisting[language=c++]{SCC_Tarjans.cpp}

    \subsection{AP \& Bridges}
    \lstinputlisting[language=c++]{AP_Bridges.cpp}

    \subsection{Network Flow}
        \subsubsection{Edmond Karp Max Flow}
        \lstinputlisting[language=c++]{Edmond_Karp.cpp}
        \subsubsection{Ford Fulkerson Max Flow}
        \lstinputlisting[language=c++]{Ford_Fulkerson.cpp}

    \newpage
    \section{Data Structures}
        \subsection{Fenwick Tree}
        \lstinputlisting[language=c++]{Fenwick.cpp}
        \subsection{UFDS}
        \lstinputlisting[language=c++]{UFDS.cpp}
        \subsection{Sparse Table}
        \lstinputlisting[language=c++]{Sparse_Table.cpp}
        \subsection{Segment Tree}
        \lstinputlisting[language=c++]{Segment_Tree.cpp}
        \subsection{Lazy Segment Tree}
        \lstinputlisting[language=c++]{Lazy_Seg_Tree.cpp}
        \subsection{Convex Hull Trick}
        \lstinputlisting[language=c++]{Convex_Hull_Trick.cpp}

    \newpage
    \section{Geometry}
        \subsection{Convex Hull}
        \lstinputlisting[language=c++]{Convex_Hull.cpp}
        \subsection{Geometry Axioms}
        \lstinputlisting[language=c++]{Geometry.cpp}

    \newpage
    \section{Strings}
        \subsection{Suffix Array}
        \lstinputlisting[language=c++]{SuffixArray.cpp}
        \subsection{Trie}
        \lstinputlisting[language=c++]{Trie.cpp}
        \subsection{KMP}
        \lstinputlisting[language=c++]{KMP.cpp}


    \newpage
    \section{Algorithms}
        \subsection{NlogN LIS}
        \lstinputlisting[language=c++]{LIS.cpp}
        \subsection{RectInHist}
        \lstinputlisting[language=c++]{RectInHist.cpp}

    \newpage
    \section{Maths}
        \subsection{Miller Rabin}
        \lstinputlisting[language=c++]{Miller_Rabin.cpp}
        \subsection{Binomial Coefficients}
        \lstinputlisting[language=c++]{binomial.cpp}
        \subsection{Gaussian Elimination}
        \lstinputlisting[language=c++]{Gaussian_Elimination.cpp}
        \subsection{Ternary Search}
        \lstinputlisting[language=c++]{Ternary_Search.cpp}
        \subsection{Matrix Exponential}
        \lstinputlisting[language=c++]{Matrix_Power.cpp}
        \subsection{Exponents}
        \lstinputlisting[language=c++]{Exponents.cpp}
        \subsection{Theorems}
            \subsubsection{Burnside's Lemma}
            Let $G$ be a finite group that acts on a set $X$. For each $g$ let $X^g$ denote the set of elements in $X$ by $g$.
            \[ X^g = \{\ x \in X \mid g.x = x \} \]
            Then the number of orbits is as follows
            \[ \lvert X/G \rvert  = \frac{1}{\lvert G \rvert} \sum_{g\in G} \lvert X^g \rvert \]
            \subsubsection{Polya's enumeration theorem}
            Let $X$ be a finite set and let $G$ be a group of permutations of $X$. Let $Y$ be a finite set of colors so that $Y^X$ is the set of colored arrangements of $X$. Then the number of orbits of $Y^X$ under $G$ is
            \[\lvert Y^X/G\rvert = \frac{1}{\lvert G \rvert} \sum_{g \in G} \lvert Y \rvert^{c(g)} \]
            where $c(g)$ is the number of cycles in permutation g.
\end{document}
